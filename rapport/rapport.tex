\documentclass[12pt,a4paper]{article}

% Packages
\usepackage[utf8]{inputenc}
\usepackage[T1]{fontenc}
\usepackage[french]{babel}
\usepackage{graphicx}
\usepackage{float}
\usepackage{booktabs}
\usepackage{amsmath}
\usepackage{hyperref}
\usepackage{geometry}
\usepackage{xcolor}
\usepackage{listings}
\usepackage{caption}
\usepackage{subcaption}

\geometry{margin=2.5cm}

% Configuration des listings
\lstset{
    language=Python,
    basicstyle=\ttfamily\small,
    keywordstyle=\color{blue},
    commentstyle=\color{green!60!black},
    stringstyle=\color{orange},
    showstringspaces=false,
    frame=single,
    breaklines=true
}

% Titre
\title{
    \vspace{-2cm}
    \textbf{Analyse des Accidents de Vélo en Île-de-France}\\
    \large Projet de Science des Données Appliquée
}
\author{
    Projet SDA\\
    \textit{Master Data Science}
}
\date{Janvier 2026}

\begin{document}

\maketitle

\begin{abstract}
Ce rapport présente une analyse approfondie des accidents de vélo en Île-de-France 
en utilisant des techniques de machine learning. Nous répondons à deux questions 
principales : (1) identifier les communes à risque élevé d'accidents, et (2) prédire 
le nombre d'accidents par commune en fonction des aménagements cyclables. L'analyse 
s'appuie sur trois sources de données fusionnées : les accidents de vélo, les 
aménagements cyclables et les comptages de trafic vélo. Les résultats montrent qu'une 
régression logistique permet de classifier le risque avec un ROC-AUC de 95.6\%, 
tandis qu'un modèle Gradient Boosting prédit le nombre d'accidents avec un R² de 89.2\%.
\end{abstract}

\tableofcontents
\newpage

% ============================================================================
\section{Introduction}
% ============================================================================

\subsection{Contexte}

La pratique du vélo connaît un essor considérable en Île-de-France ces dernières années,
favorisée par les politiques de mobilité durable et le développement des infrastructures
cyclables. Cependant, cette augmentation du trafic cycliste s'accompagne d'une
problématique de sécurité routière qu'il convient d'analyser et de modéliser.

\subsection{Objectifs du Projet}

Ce projet vise à répondre à deux questions fondamentales :

\begin{enumerate}
    \item \textbf{Classification du risque} : La commune présente-t-elle un risque 
    élevé d'accidents de vélo ?
    \item \textbf{Prédiction du nombre d'accidents} : Peut-on prédire le nombre 
    d'accidents dans une commune en fonction de ses aménagements cyclables ?
\end{enumerate}

\subsection{Méthodologie}

Notre approche se décompose en plusieurs étapes :
\begin{itemize}
    \item Collecte et fusion de trois sources de données
    \item Préparation et nettoyage des données
    \item Analyse exploratoire
    \item Modélisation par machine learning
    \item Comparaison des algorithmes et conclusions
\end{itemize}

% ============================================================================
\section{Description des Données}
% ============================================================================

\subsection{Sources de Données}

Nous utilisons trois jeux de données complémentaires :

\subsubsection{Accidents de Vélo}
Ce dataset contient \textbf{80 022 accidents} sur l'ensemble de la France, dont 
\textbf{22 609 en Île-de-France}. Les principales variables sont :
\begin{itemize}
    \item Localisation (département, commune, coordonnées GPS)
    \item Date et heure de l'accident
    \item Gravité (indemne, blessé léger, hospitalisé, tué)
    \item Conditions (luminosité, météo, état de la chaussée)
    \item Caractéristiques de la victime (âge, sexe)
\end{itemize}

\subsubsection{Aménagements Cyclables}
Ce dataset recense \textbf{143 060 aménagements cyclables} en Île-de-France :
\begin{itemize}
    \item Type de voie (piste cyclable, bande cyclable, voie partagée)
    \item Longueur de l'aménagement
    \item Revêtement et état
    \item Localisation par commune (code INSEE)
\end{itemize}

\subsubsection{Comptages Vélo}
Ce dataset contient \textbf{933 757 mesures} de comptage horaire :
\begin{itemize}
    \item Identifiant et localisation du compteur
    \item Comptage horaire de passages
    \item Date et heure de la mesure
\end{itemize}

\subsection{Fusion des Données}

Les données ont été fusionnées au niveau communal (code INSEE) pour créer un 
dataset final de \textbf{1 124 communes} avec 38 variables. La Figure~\ref{fig:pipeline}
illustre le processus de fusion.

\begin{figure}[H]
    \centering
    \fbox{\parbox{0.9\textwidth}{
        \centering
        \textbf{Pipeline de Fusion des Données}\\[0.5cm]
        Accidents (22 609) $\rightarrow$ Agrégation par commune $\rightarrow$ 770 communes\\
        $\downarrow$\\
        Aménagements (143 060) $\rightarrow$ Agrégation par commune $\rightarrow$ 1 053 communes\\
        $\downarrow$\\
        Fusion (outer join sur code INSEE)\\
        $\downarrow$\\
        \textbf{Dataset Final : 1 124 communes, 38 features}
    }}
    \caption{Pipeline de préparation des données}
    \label{fig:pipeline}
\end{figure}

\subsection{Variables Créées}

Plusieurs features ont été calculées lors de la fusion :
\begin{itemize}
    \item \texttt{nb\_accidents} : nombre total d'accidents par commune
    \item \texttt{nb\_accidents\_graves} : accidents avec hospitalisation ou décès
    \item \texttt{taux\_accidents\_graves} : ratio d'accidents graves
    \item \texttt{nb\_amenagements} : nombre d'infrastructures cyclables
    \item \texttt{longueur\_totale\_amenagements} : longueur cumulée en mètres
    \item \texttt{ratio\_pistes\_cyclables} : proportion de pistes séparées
    \item \texttt{risque\_eleve} : variable binaire (1 si $\geq$ 75\textsuperscript{e} percentile)
\end{itemize}

% ============================================================================
\section{Analyse 1 : Classification du Risque}
% ============================================================================

\subsection{Problématique}

La première question consiste à classifier les communes selon leur niveau de risque
d'accidents de vélo. On définit une commune à \textbf{risque élevé} si son nombre
d'accidents est supérieur ou égal au 75\textsuperscript{e} percentile (soit $\geq 6$ accidents).

\subsection{Modèles Testés}

Six algorithmes de classification ont été comparés :
\begin{enumerate}
    \item Régression Logistique (avec pondération des classes)
    \item Random Forest Classifier
    \item Gradient Boosting Classifier
    \item XGBoost Classifier
    \item LightGBM Classifier
    \item SVM avec noyau RBF
\end{enumerate}

\subsection{Résultats}

\begin{table}[H]
    \centering
    \caption{Comparaison des modèles de classification}
    \label{tab:classification}
    \begin{tabular}{lccccc}
        \toprule
        \textbf{Modèle} & \textbf{Accuracy} & \textbf{Precision} & \textbf{Recall} & \textbf{F1-Score} & \textbf{ROC-AUC} \\
        \midrule
        Logistic Regression & 0.884 & 0.758 & 0.833 & \textbf{0.794} & \textbf{0.956} \\
        Random Forest & 0.898 & 0.863 & 0.733 & 0.793 & 0.951 \\
        SVM (RBF) & 0.880 & 0.739 & 0.850 & 0.791 & 0.953 \\
        LightGBM & 0.889 & 0.807 & 0.767 & 0.786 & 0.940 \\
        XGBoost & 0.884 & 0.793 & 0.767 & 0.780 & 0.941 \\
        Gradient Boosting & 0.880 & 0.824 & 0.700 & 0.757 & 0.953 \\
        \bottomrule
    \end{tabular}
\end{table}

\subsection{Analyse des Résultats}

Le modèle de \textbf{Régression Logistique} obtient les meilleures performances 
globales avec :
\begin{itemize}
    \item Un \textbf{F1-Score de 79.4\%}, équilibrant precision et recall
    \item Un \textbf{ROC-AUC de 95.6\%}, indiquant une excellente capacité discriminative
    \item Une bonne stabilité en validation croisée (CV F1 = 0.816 $\pm$ 0.033)
\end{itemize}

La Figure~\ref{fig:roc} présente les courbes ROC comparatives des modèles.

\begin{figure}[H]
    \centering
    \includegraphics[width=0.8\textwidth]{../outputs/roc_curves_classification.png}
    \caption{Courbes ROC des modèles de classification}
    \label{fig:roc}
\end{figure}

\subsection{Importance des Features}

L'analyse de l'importance des features révèle que les variables les plus 
discriminantes sont :
\begin{enumerate}
    \item \texttt{nb\_amenagements} : nombre d'infrastructures cyclables
    \item \texttt{longueur\_totale\_amenagements} : longueur des aménagements
    \item \texttt{nb\_pistes\_cyclables} : nombre de pistes dédiées
    \item \texttt{est\_paris} : indicateur d'arrondissement parisien
\end{enumerate}

\begin{figure}[H]
    \centering
    \includegraphics[width=0.9\textwidth]{../outputs/feature_importance_classification.png}
    \caption{Importance des features par modèle (Classification)}
    \label{fig:feat_class}
\end{figure}

% ============================================================================
\section{Analyse 2 : Prédiction du Nombre d'Accidents}
% ============================================================================

\subsection{Problématique}

La seconde question vise à prédire le nombre exact d'accidents par commune en 
fonction des caractéristiques des aménagements cyclables. Il s'agit d'un problème 
de \textbf{régression}.

\subsection{Modèles Testés}

Sept algorithmes de régression ont été comparés :
\begin{enumerate}
    \item Régression Linéaire
    \item Ridge Regression (régularisation L2)
    \item Lasso Regression (régularisation L1)
    \item Random Forest Regressor
    \item Gradient Boosting Regressor
    \item XGBoost Regressor
    \item LightGBM Regressor
\end{enumerate}

\subsection{Résultats}

\begin{table}[H]
    \centering
    \caption{Comparaison des modèles de régression}
    \label{tab:regression}
    \begin{tabular}{lcccc}
        \toprule
        \textbf{Modèle} & \textbf{RMSE} & \textbf{MAE} & \textbf{R²} & \textbf{CV R²} \\
        \midrule
        Gradient Boosting & \textbf{37.06} & \textbf{8.40} & \textbf{0.892} & 0.897 $\pm$ 0.025 \\
        XGBoost & 37.53 & 8.43 & 0.889 & \textbf{0.900 $\pm$ 0.018} \\
        Random Forest & 37.80 & 9.06 & 0.888 & 0.886 $\pm$ 0.027 \\
        Linear Regression & 42.89 & 11.39 & 0.856 & 0.855 $\pm$ 0.072 \\
        Ridge Regression & 43.23 & 11.34 & 0.853 & 0.857 $\pm$ 0.067 \\
        Lasso Regression & 43.87 & 11.38 & 0.849 & 0.856 $\pm$ 0.065 \\
        LightGBM & 45.41 & 10.29 & 0.838 & 0.825 $\pm$ 0.074 \\
        \bottomrule
    \end{tabular}
\end{table}

\subsection{Analyse des Résultats}

Le modèle \textbf{Gradient Boosting} obtient les meilleures performances :
\begin{itemize}
    \item \textbf{R² = 0.892} : le modèle explique 89.2\% de la variance
    \item \textbf{MAE = 8.40} : erreur moyenne de 8.4 accidents par commune
    \item \textbf{RMSE = 37.06} : pénalise davantage les grosses erreurs
    \item Excellente stabilité en validation croisée
\end{itemize}

\begin{figure}[H]
    \centering
    \includegraphics[width=0.9\textwidth]{../outputs/predictions_vs_reel.png}
    \caption{Prédictions vs valeurs réelles pour différents modèles}
    \label{fig:pred}
\end{figure}

\subsection{Analyse des Corrélations}

\begin{figure}[H]
    \centering
    \includegraphics[width=0.8\textwidth]{../outputs/correlation_features.png}
    \caption{Corrélation des features avec le nombre d'accidents}
    \label{fig:corr}
\end{figure}

Les features les plus corrélées avec le nombre d'accidents sont :
\begin{itemize}
    \item \texttt{nb\_pistes\_cyclables} : $r = 0.844$
    \item \texttt{nb\_amenagements} : $r = 0.841$
    \item \texttt{longueur\_totale\_amenagements} : $r = 0.817$
\end{itemize}

\textbf{Interprétation} : La corrélation positive entre aménagements et accidents 
s'explique par le fait que les communes bien équipées ont aussi un trafic cycliste 
plus important, donc statistiquement plus d'accidents.

% ============================================================================
\section{Discussion}
% ============================================================================

\subsection{Synthèse des Résultats}

\begin{table}[H]
    \centering
    \caption{Synthèse des meilleurs modèles}
    \begin{tabular}{lll}
        \toprule
        \textbf{Question} & \textbf{Meilleur Modèle} & \textbf{Performance} \\
        \midrule
        Classification du risque & Régression Logistique & ROC-AUC = 95.6\% \\
        Prédiction nb accidents & Gradient Boosting & R² = 89.2\% \\
        \bottomrule
    \end{tabular}
\end{table}

\subsection{Limites de l'Étude}

\begin{itemize}
    \item \textbf{Biais de report} : tous les accidents ne sont pas déclarés
    \item \textbf{Données de comptage} : seuls 69 compteurs pour toute l'IDF
    \item \textbf{Variables manquantes} : population, densité urbaine, trafic automobile
    \item \textbf{Temporalité} : pas de prise en compte de l'évolution temporelle
\end{itemize}

\subsection{Perspectives}

Pour améliorer les modèles, on pourrait :
\begin{itemize}
    \item Intégrer des données démographiques (population, densité)
    \item Ajouter des données de trafic automobile
    \item Effectuer une analyse temporelle (tendances, saisonnalité)
    \item Utiliser des modèles spatiaux (autocorrélation géographique)
\end{itemize}

% ============================================================================
\section{Conclusion}
% ============================================================================

Ce projet a permis de développer deux modèles prédictifs performants pour l'analyse 
des accidents de vélo en Île-de-France :

\begin{enumerate}
    \item Un \textbf{modèle de classification} (Régression Logistique) capable 
    d'identifier les communes à risque élevé avec une précision de 88\% et un 
    ROC-AUC de 96\%.
    
    \item Un \textbf{modèle de régression} (Gradient Boosting) capable de prédire 
    le nombre d'accidents avec un R² de 89\%, soit une erreur moyenne de 8.4 
    accidents par commune.
\end{enumerate}

\textbf{Recommandations pour les décideurs :}
\begin{itemize}
    \item Prioriser les communes à haut risque identifiées par le modèle
    \item Privilégier les pistes cyclables séparées de la circulation
    \item Améliorer l'éclairage et la signalisation sur les voies cyclables
    \item Développer les comptages pour mieux mesurer l'exposition au risque
\end{itemize}

Les modèles développés constituent un outil d'aide à la décision pour les 
collectivités souhaitant optimiser leur politique de sécurité cycliste.

% ============================================================================
\section*{Annexes}
% ============================================================================

\subsection*{A. Structure du Projet}

\begin{verbatim}
Projet_sda_top/
+-- data/
|   +-- accidentsVelo.csv
|   +-- amenagements-velo-en-ile-de-france.csv
|   +-- comptage-velo-donnees-compteurs.csv
|   +-- dataset_final_idf.csv
+-- src/
|   +-- 01_preparation_donnees.py
|   +-- 02_classification_risque.py
|   +-- 03_prediction_accidents.py
+-- outputs/
|   +-- roc_curves_classification.png
|   +-- confusion_matrix_classification.png
|   +-- feature_importance_classification.png
|   +-- predictions_vs_reel.png
|   +-- ...
+-- rapport/
|   +-- rapport.tex
+-- requirements.txt
\end{verbatim}

\subsection*{B. Technologies Utilisées}

\begin{itemize}
    \item \textbf{Python 3.12}
    \item \textbf{pandas} : manipulation de données
    \item \textbf{scikit-learn} : modèles de machine learning
    \item \textbf{XGBoost, LightGBM} : modèles de gradient boosting
    \item \textbf{matplotlib, seaborn} : visualisation
    \item \textbf{LaTeX} : rédaction du rapport
\end{itemize}

\end{document}
